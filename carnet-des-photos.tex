% This is a template for creating a 'carnet des photos' with LaTeX. 
% The design of this template is minimalist, with a focus on images.
% There are no indexes, table of contents, numbered pages.
% Captions on images are optional; if present, the color is gray to minimize contrast and focus on the images.
% Chapters are not numbered, and can be used to divide the book in multiple stories.

% It was created by Fabio Sirna (https://fabiosirna.com), in September 2023 with the goal of being integrated 
% into a FOSS photo workflow specifically using digikam (DAM), darktable (editing) and LaTeX (publishing).
%

%
% By default the  
%


\documentclass[a4paper,twoside]{book}
\usepackage{geometry}
\geometry{
	layout=a4paper,
	ignoreall,
	top=7em,
	bottom=7em,
	left=7em,
	right=7em,
	bindingoffset=5em
}

%
% Uncomment the next lines to create a carnet in A5 format 
%

% \documentclass[a5paper,twoside]{book}

% \usepackage{geometry}
% \geometry{
% 	layout=a5paper,
% 	ignoreall,
% 	top=5em,
% 	bottom=5em,
% 	left=5em,
% 	right=5em,
% 	bindingoffset=2em
% }

\usepackage[utf8]{inputenc}
\usepackage[english]{babel} %Change or add other languages separated by a comma
\usepackage{calc} % Package to calculate dimensions
\usepackage{float}

\usepackage{parskip}

%
% I use the geometry package to set the carnet format:
%
% ignoreall -> ignore header, footer and margins
% bindingoffet -> Adds a space in the book spine to consider binding
%

%
% Use the graphicx package to import images and set the default path where to find them
%

\usepackage{graphicx}
\graphicspath{ {images/} }

%
% Removes image caption numbering  
%

\usepackage{caption}
\captionsetup{labelformat=empty}

%
% I'm using the Fira Sans font for the typography of the book. 
% Change the font to another one choosing the typeface from https://tug.org/FontCatalogue/
%

\usepackage[sfdefault,book]{FiraSans} %% option 'sfdefault' activates Fira Sans as the default text font
\usepackage[T1]{fontenc}
\renewcommand*\oldstylenums[1]{{\firaoldstyle #1}}

%
% I'm using the fancyhdr package to hide the pages numbers
%

\usepackage{fancyhdr}

\pagestyle{fancy}
\fancyhf{}
\renewcommand{\headrulewidth}{0pt}
\fancyfoot{}

%
% I'm using titlesec package to customize the style of chapter titles
%

\usepackage{titlesec}  

\titleformat{\chapter}[display]
  {\normalfont\Huge\bfseries\raggedleft}
  {}
  {0pt}
  {\vfill\MakeUppercase} % Chapter are uppercase
  [\vfill\thispagestyle{empty}] % Remove page number

%
% Change the color of the text
%
\usepackage[dvipsnames]{xcolor}

% Definisci il colore nero al 90%
% \definecolor{black90}{rgb}{0,0,0.5}

% Set the default text color
\color{Gray}

%
% Uncomment next line to use the package and print the layout design. 
% See section at the end of this template.
%
% \usepackage{layout}  
%

%
% \usepackage{lipsum}
%

% Inserts a blank page
\newcommand{\blankpage}{\newpage\hbox{}\thispagestyle{empty}\newpage}

%%%%%%%%%%%%%%%%%%%%%
% Begin of document %
%%%%%%%%%%%%%%%%%%%%%

\begin{document}

% To print current layout

\frontmatter 

\color{Black}
\title{Carnet des photo}
\author{Fabio Sirna}
\date{September 2023}

\maketitle

\mainmatter 

%
% I set up a page to insert a book dedication
%

\thispagestyle{empty} % Hide page number

\begin{flushright}
    \vspace*{\stretch{1}} % Blank space that takes up all the rest of the page
    \emph{Dedicated to anyone}\\
    \emph{you want to dedicate this page.}\\
    \vspace*{\stretch{2}} % Blank space to push text to the upper right.
\end{flushright}

% First chapter
% Color of the chapter can be customised to match —for example— the color of the cover.
% Chapter is not numbered, remove asterisk to add automatic numbering

\color{BrickRed}
\chapter*{Chapter title}

% After the chapter page, I like to have a blank page to start the story on an odd page. 
\blankpage

\clearpage

\begin{figure}[H]
	\includegraphics[width=\textwidth]{landscape.png}
	\caption{My beautiful caption that can probably span more lines, as it can be very very long because I don't know what I'm going to write here as it depends from my creativity and information I want to write here.}
\end{figure}

\clearpage

\begin{figure}[H]
	\includegraphics[width=\textwidth]{landscape.png}
	\caption{My beautiful caption that can probably span more lines, as it can be very very long because I don't know what I'm going to write here as it depends from my creativity and information I want to write here.}
\end{figure}

\clearpage

\begin{figure}[H]
	\includegraphics[width=\textwidth]{portrait.png}
	\caption{My beautiful caption that can probably span more lines, as it can be very very long because I don't know what I'm going to write here as it depends from my creativity and information I want to write here.}
\end{figure}

\clearpage

\begin{figure}[H]
	\includegraphics[width=\textwidth]{portrait.png}
	\caption{My beautiful caption that can probably span more lines, as it can be very very long because I don't know what I'm going to write here as it depends from my creativity and information I want to write here.}
\end{figure}

\clearpage

\begin{figure}[H]
	\includegraphics[width=\textwidth]{portrait.png}
	\caption{My beautiful caption that can probably span more lines, as it can be very very long because I don't know what I'm going to write here as it depends from my creativity and information I want to write here.}
\end{figure}

\clearpage

\begin{figure}[H]
	\includegraphics[width=\textwidth]{portrait.png}
	\caption{My beautiful caption that can probably span more lines, as it can be very very long because I don't know what I'm going to write here as it depends from my creativity and information I want to write here.}
\end{figure}

\clearpage 

\begin{figure}[H]
	\includegraphics[width=\textwidth]{landscape.png}
	\caption{My beautiful caption that can probably span more lines, as it can be very very long because I don't know what I'm going to write here as it depends from my creativity and information I want to write here.}
\end{figure}

\clearpage

\begin{figure}[H]
	\includegraphics[width=\textwidth]{landscape.png}
	\caption{My beautiful caption that can probably span more lines, as it can be very very long because I don't know what I'm going to write here as it depends from my creativity and information I want to write here.}
\end{figure}

\clearpage

\begin{figure}[H]
	\includegraphics[width=\textwidth]{landscape.png}
	\caption{My beautiful caption that can probably span more lines, as it can be very very long because I don't know what I'm going to write here as it depends from my creativity and information I want to write here.}
\end{figure}

\clearpage

\begin{figure}[H]
	\includegraphics[width=\textwidth]{landscape.png}
	\caption{My beautiful caption that can probably span more lines, as it can be very very long because I don't know what I'm going to write here as it depends from my creativity and information I want to write here.}
\end{figure}

% To insert a blank page in case the number of pictures is odd and have the next chapter start on an odd page.

\blankpage

% Other chapters

\chapter*{Yet another chapter}

% After the chapter page, I like to have a blank page to start the story on an odd page. 
\blankpage

\clearpage

\begin{figure}[H]
	\includegraphics[width=\textwidth]{landscape.png}
	\caption{My beautiful caption that can probably span more lines, as it can be very very long because I don't know what I'm going to write here as it depends from my creativity and information I want to write here.}
\end{figure}

\clearpage

\begin{figure}[H]
	\includegraphics[width=\textwidth]{portrait.png}
	\caption{My beautiful caption that can probably span more lines, as it can be very very long because I don't know what I'm going to write here as it depends from my creativity and information I want to write here.}
\end{figure}

\clearpage

\begin{figure}[H]
	\includegraphics[width=\textwidth]{landscape.png}
	\caption{My beautiful caption that can probably span more lines, as it can be very very long because I don't know what I'm going to write here as it depends from my creativity and information I want to write here.}
\end{figure}

\blankpage

% If you want to change the design of the template remove the comment in the following section to print the layout package information.
% See the comment at the beginning of the file. 

% \chapter*{Layout}
% \newpage
% \layout

\end{document}